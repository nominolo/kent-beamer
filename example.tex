\documentclass[sans,mathserif]{beamer}

\usepackage{graphicx} % graphics
\usepackage{epsfig} % eps graphics
\usepackage{hyperref} % urls
\usepackage{booktabs, caption} % table styling

% suppress navigation bar
\beamertemplatenavigationsymbolsempty

\usepackage{fouriernc}
\usepackage[T1]{fontenc}
\usepackage{helvet}
%% \usepackage{cmbright}
\usepackage{eulervm}

%% \usefonttheme{serif}


\mode<presentation>
{
  \usetheme{kent}
  \setbeamercovered{transparent}
  \setbeamertemplate{items}[circle]
}

\title{A Survey of Elements in Whoville}
\author{Samantha Sample \\ \texttt{sam@example.com}}
\institute[The Institute of Advanced Study]
{
Department of Chemistry \\
Whoville College \\
}

\date{\today}

\begin{document}

\section{Introduction}

%% -------------------------------------------------------------------

\begin{frame}
  \titlepage
\end{frame}

%% -------------------------------------------------------------------

\begin{frame}{A sample slide}

A displayed formula:

\[
  \int_{-\infty}^\infty e^{-x^2} \, dx = \sqrt{\pi}
\]

An itemized list:

\begin{itemize}
  \item itemized item 1
  \item itemized item 2
  \item itemized item 3
\end{itemize}

\begin{theorem}
  In a right triangle, the square of hypotenuse equals
  the sum of squares of two other sides.
\end{theorem}

\end{frame}

%% -------------------------------------------------------------------

\begin{frame}{Theorems and such}

\begin{definition}
  A triangle that has a right angle is called
  a \emph{right triangle}.
\end{definition}

\begin{theorem}
  In a right triangle, the square of the hypotenuse
  equals the sum of the squares of the two other sides.
\end{theorem}

\begin{proof}
  We leave the proof as an exercise to our astute reader.
  We also suggest that the reader generalize the proof to
  non-Euclidean geometries.
\end{proof}

\end{frame}

%% -------------------------------------------------------------------

\begin{frame}{Outline of the talk} 
 
\begin{itemize} 
  \item Introduction 
  \pause 
  \item Statement of the main theorem 
  \pause 
  \item Technical lemmata 
  \pause 
  \item Proof of the main theorem 
  \pause 
  \item Conclusions 
\end{itemize} 

\end{frame}

%% -------------------------------------------------------------------

\begin{frame}{Fermat's Last Theorem} 
 
In this talk I will give a very elementary proof of the 
theorem.  I am surprised that no one else has thought of 
this before. 
\medskip 
 
\pause 
 
Fermat's Last Theorem says that the equation 
\[ 
  x^2 + y^2 = z^2 
\] 
has no solution in the set of natural numbers. 
\medskip 
 
\pause 
 
This is not true.  After a lengthy calculation on the 
department's Linux machines, I have verified that within 
the numerical accuracy of the Pentium-4 processor, we have: 
\[ 
  5000^2 + 12000^2 = 13000^2 
\] 
 
\end{frame}

%% -------------------------------------------------------------------

%% -------------------------------------------------------------------

%% -------------------------------------------------------------------



\end{document}
